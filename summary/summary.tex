\chapter*{Abstract}

This paper presents a study on browser-based port scanning, a technique that allows for the detection of open ports on a target system through the use of a web browser. The research investigates the optimal strategy for browser-based port scanning, the feasibility of using port scanning to identify specific programs running on a user's system, and the uniqueness of browser-based port scanning fingerprints.
The study demonstrates that browser-based port scanning can serve as an effective alternative to traditional port scanning techniques. The results suggest that browser-based port scanning can accurately identify specific programs running on a user's system. This has concerning implications because browser-based port scanning is a client-side, local operation on the user's system, unlike regular port scanning, which might be leveraged to bypass intrusion detection systems, such as a firewall.
Furthermore, the study estimates the uniqueness of browser-based port scanning fingerprints, which has significant implications for user privacy and internet anonymity. The study reveals that browser-based port scanning fingerprints are distinct enough to be employed as a means of tracking users across various websites, highlighting the need for enhanced privacy measures by modern web browsers.  