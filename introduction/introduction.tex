\chapter{Introduction}
% You can start your report by copying the introduction from your VAF. The advice is to rewrite the introduction \emph{after} you did write the remainder of the thesis.

% In the introduction, you:

% \begin{itemize}
% 	\item motivate why this research is important. You can bring forward motivation with respect to society, or show that your research is scientifically interesting; preferably both
% 	\item give some background information
% 	\item describe the goal of your research
% 	\item give the reader an overview of the structure of the remainder of your thesis.
% \end{itemize}

% This sentence is only here to show you how to refer to a source~\citep{Dijkstra-1968}.

% \begin{table}[h!tbp]
% \begin{tabular}{l | r | r| c}
% kolom 1 & kolom 2 & kolom 3 & kolom 4 \\
% \hline
% zon & maan & ster & meteoor\\
% gras & graan & groen & grauw\\
% \end{tabular}
% \caption{Example table}
% \label{table-example}
% \end{table}

% Table~\ref{table-example} shows how to include a table. Note that the first column is left-justified, the right column is centered, and the other two columns are right-justified (because of the \texttt{\{l | r | r | c\}}). More information: \url{https://en.wikibooks.org/wiki/LaTeX/Tables}. 

% \texttt{[h!tb]} means: preferably place the table \emph{h}ere, and if that is not possible, at the \emph{t}op of the page, at the \emph{b}ottom, or on a separate \emph{page}. The same positioning advice can be used in figures. Figure~\ref{fig-example} is an example.

% \begin{figure}[h!tbp]
% \includegraphics[scale=0.5]{LaTeX.png}
% \caption{LaTeX}
% \label{fig-example}
% \end{figure}

% The following chapters are an example of how you could structure your thesis. Do not hesitate to use a different structure!

As the world becomes more digitized, online privacy and user tracking have become major concerns. Websites are collecting vast amounts of data about their visitors, including sensitive information about their devices and browsing behavior. Websites often collect user data, such as search queries, IP addresses, and click behavior, through tracking technologies like cookies and web beacons. The collection of this data can be used for various purposes, such as targeted advertising, personalization, and user profiling.
 
Several studies have explored the implications of user tracking and privacy. Acar et al.~\citescientific{acar2014} found that a significant number of websites are capable of tracking users across multiple visits, posing a potential threat to user privacy. In a similar vein, Mayer and \\ Mitchell~\citescientific{mayer2012} have examined the privacy implications of third-party tracking on the web and have proposed measures for policymakers and developers to address these concerns.
Furthermore, the privacy risks associated with personalized advertising have been analyzed by Komanduri et al.~\citescientific{komanduri2011}, who argue that the collection of user data for this purpose can result in significant privacy risks, since users are often unaware of what data is being collected and how it is being used. Similarly, McDonald et al.~\citescientific{mcdonald2009} have investigated the effectiveness of privacy notices and have found that users frequently do not comprehend the information presented in these notices or the implications of data collection and sharing. These studies and others highlight the importance of protecting user privacy in the context of user tracking and data collection on the web.

Seemingly harmless information, such as browsing habits or search queries, can reveal personal details like interests, location, and even sensitive information like health conditions or financial status. This data is often used for targeted advertising, personalized content, and profiling individuals. However, uncontrolled data collection raises ethical concerns, such as lack of transparency and user control over personal information. Moreover, some tracking techniques may be unlawful in certain countries, violating data protection and privacy laws that impose legal obligations on organizations. Dismissing privacy concerns with the argument `it does not matter' is flawed, as it ignores the potential risks and consequences associated with unregulated data collection and use by online entities. Prioritizing user privacy, promoting transparency, and complying with applicable laws are imperative in safeguarding individuals' privacy on the web.

One popular technique used for user tracking and profiling is browser fingerprinting. It involves collecting various information from a user's browser, such as the user agent string, screen resolution, installed fonts, and plugins. This information can be used to create a unique identifier or \emph{fingerprint} of the user's browser, which can then be used to track the user across different websites and browsing sessions. Cookies, along with browser fingerprinting, are commonly utilized for tracking purposes and are a widely used method for monitoring user activities. They are explicitly referenced in the General Data Protection Regulation (GDPR), which is Europe's largest privacy law~\citeregulatory{gdpr}.

A browser tracking technique that has not been extensively researched is port scanning. Port scanning is commonly used as a network security technique that involves scanning a network for open ports to identify potential vulnerabilities. While not commonly used for user tracking, port scanning can potentially reveal information about a user's device, operating system, and running programs. An important distinction to make with cookies, is that port scanning does not require user consent and is therefore a less transparent tracking technique.
In a high-profile case in 2020~\citearticle{forbes_ebay,ebay_port_scans}, popular ecommerce website eBay used port scanning as a security measure to identify remote access tools on users' systems. Many users were infected with malware at the time, and attackers were using compromised computers to make purchases on the website. eBay used port scanning to detect these remote access tools and prevent malicious users from making purchases. However, security and privacy experts were critical of this security implementation, as scanning the local network has implications for both security and privacy.

The focus of this research will be on port scanning via the browser. There are 65,536 ports provided by the TCP/IP protocol for an IP address in the computer. Among them, the range of well known ports is from 0 to 1023, the range of registered ports is from 1024 to 49,151, and the range of dynamic ports is from 49,152 to 65,535~\citescientific{yuan2020}.
This wide range of ports makes it an interesting topic for research, as each port can reveal sensitive information about a system. Browser-based port scanning has the potential to enhance existing browser fingerprinting methods.
It is important to differentiate between port scanning in general and port scanning through a browser. 
Scanning ports through a browser is less intrusive and more difficult to detect by intrusion detection systems. This is because port scanning through a browser involves making regular network requests, rather than directly sending probes at the protocol level. Furthermore, a website can passively perform port scanning without requiring user privileges to do so.
Many port scanners have been developed, such as Advanced Port Scanner~\citescientific{el2011}, SATAN~\citescientific{arce2008}, Angry IP Scanner~\citescientific{el2011}, and the most popular open source solution, Nmap~\citescientific{orebaugh201}. However, all of these port scanners focus on port scanning at the protocol level, and they are not limited by abstraction layers that exist in a browser environment. JavaScript APIs do not have access to TCP sockets directly, unlike scanning tools such as Nmap. This severely limits the techniques that can be used to detect open ports. Consequently, the methods used for local port scanning from a browser environment are significantly different from those used by regular scanning tools.

Despite the clear distinction between browser-based port scanning and regular scanning tools, there is limited research on the former. The difference is not just technical, but also relates to the target audience for the scans. Regular scanning tools like Nmap are proactive, whereas websites could use port scanning to passively target their visitors. For instance, a website targeting specific users could scan for particular ports that might reveal sensitive information about this type of user.
Port scanning from the browser can have both positive and negative purposes. It can be used as a security measure to identify remote access tools, or used with malicious intent to gather information about users. However, due to its potential impact on online privacy and security, it remains an essential area for research, especially because these port scans can be done without the consent of the users.
The field of browser-based port scanning has received limited attention in the literature, leaving many intriguing questions regarding its potential impact on privacy. This research focuses on the unexplored use of browser-based port scanning as a browser tracking technique and evaluating its potential implications for user privacy. 

The research has three major contributions:
\begin{itemize}
    \item Estimating the most effective and efficient scanning technique: Before we can assess the privacy risks of browser-based port scanning, it is important to find the most effective scanning technique, efficiency is also a crucial consideration, as port scanning is a time-intensive process. Finding the optimal balance between efficiency and efficacy is critical.
    \item What information browser-based port scanning can reveal about a user: Here we dive into the potential privacy risks of browser-based port scanning, and investigate what information can be revealed through browser-based port scanning.
    \item Estimating the uniqueness of browser-based port scanning fingerprints: Lastly, we estimate the entropy of browser-based port scanning. This contributes to existing browser fingerprinting research by estimating the risk to user anonymity on the internet. 
\end{itemize}

\subsubsection{Thesis overview}

This paper is structured as follows. Firstly, the research questions are discussed, with the primary research question being: \emph{What information can websites extract from clients via browser-based port scanning?} 

Following that, the background chapter explains the basics of port scanning and its ethical and legal implications. Chapter 4 describes related work and  provides an overview of existing literature on browser fingerprinting techniques, as well as related port scanning research.
Subsequently, the study investigates the optimal strategy for browser-based port scanning in Chapter 5, considering efficacy and efficiency. Furthermore, Chapter 6 focuses on the feasibility of using port scanning to identify specific programs running on a user's system, analyzing the privacy implications of this approach, and assessing its effectiveness as a means of tracking users. 
Chapter 7 estimates the uniqueness of browser-based port scanning fingerprints.
Lastly, Chapter 8 concludes the research based on the answers to the research questions, as well as discussing future work.