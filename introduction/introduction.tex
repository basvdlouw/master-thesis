\chapter{Introduction}
You can start your report by copying the introduction from your VAF. The advice is to rewrite the introduction \emph{after} you did write the remainder of the thesis.

In the introduction, you:

\begin{itemize}
	\item motivate why this research is important. You can bring forward motivation with respect to society, or show that your research is scientifically interesting; preferrably both
	\item give some background information
	\item describe the goal of your research
	\item give the reader an overview of the structure of the remainder of your thesis.
\end{itemize}

This sentence is only here to show you how to refer to a source~\citep{Dijkstra-1968}.

\begin{table}[h!tbp]
\begin{tabular}{l | r | r| c}
kolom 1 & kolom 2 & kolom 3 & kolom 4 \\
\hline
zon & maan & ster & meteoor\\
gras & graan & groen & grauw\\
\end{tabular}
\caption{Example table}
\label{table-example}
\end{table}

Table~\ref{table-example} shows how to include a table. Note that the first column is left-justified, the right column is centered, and the other two columns are right-justified (because of the \texttt{\{l | r | r | c\}}). More information: \url{https://en.wikibooks.org/wiki/LaTeX/Tables}. 

\texttt{[h!tb]} means: preferrably place the table \emph{h}ere, and if that is not possible, at the \emph{t}op of the page, at the \emph{b}ottom, or on a separate \emph{page}. The same positioning advice can be used in figures. Figure~\ref{fig-example} is an example.

\begin{figure}[h!tbp]
\includegraphics[scale=0.5]{LaTeX.png}
\caption{LaTeX}
\label{fig-example}
\end{figure}

The following chapters are an example of how you could structure your thesis. Do not hesitate to use a different structure!