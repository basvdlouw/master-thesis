\chapter{Conclusions and Discussion}

In this document, we researched browser-based port scanning and drew the following conclusions based on our findings: 
\\

\textbf{RQ1: How to choose the optimal port-scanning technique for a specific victim client in combination with a specific attack goal?}

We measured significant variations between different operating systems and scanning techniques. The Fetch API emerged as the most effective JavaScript API for conducting browser-based port scanning, and it proved to be highly efficient across various configurations. This can primarily be attributed to the \emph{no-cors} mode, a setting unique to the Fetch API.
A notable contrast was observed between Ubuntu and Windows. Ubuntu tended to time out network requests whenever possible, whereas Windows adhered to the configured socket timeout settings. This characteristic made browser-based port scanning considerably more efficient on Ubuntu.

We estimate that on Windows/Chrome, which is the most popular combination of OS/Browser, it is possible to scan 1,000 ports within one second (2,500+ on Ubuntu). This makes browser-based port scanning a practical real-world attack on virtually any web page. This is especially concerning because the scanning process can continue throughout a user's entire session on the website, which usually lasts much longer than just one second. 
\\

\textbf{RQ2: What information can browser-based port scanning reveal about the underlying operating system?}

We found that operating systems by default do not expose ports that can be revealed through a browser, with the exception of CUPS on Ubuntu. Most OS ports fall under the restricted port range and cannot be detected through browser-based port scanning. However, both Windows and Ubuntu host a \texttt{/etc/services} file which contains a mapping between ports and OS services. This mapping can be used as a foundation for a targeted scan to fingerprint the underlying OS.
\\

\textbf{RQ3: What information can browser-based port scanning reveal about specific programs
running locally on a user’s system?}

Unlike OS fingerprinting, fingerprinting specific programs is more effective. 
We found many programs from different categories that open a distinct port on the local system, which can be revealed through browser-based port scanning. 
We argue that browser-based port scanning can be used to track users by identifying open ports corresponding to specific applications, and is thereby a direct threat to user privacy/anonymity on the internet, especially when combined with known fingerprinting techniques.

When we can successfully identify open ports and the associated programs, it provides a detailed user profile that can be used for a multitude of purposes. 
These can include personalized web pages or more nefarious use cases such as phishing attempts.
Additionally, this information could be exploited to scan for known exploits of specific programs. 
In essence, the knowledge gained from these open ports not only undermines user privacy, but also establishes a foundation for a range of potentially malicious activities.
\\

\textbf{RQ4: How unique are browser-based port scanning fingerprints?}

We estimate that browser-based port scanning results in highly unique fingerprints, even with a very limited scanning time of only one second. We emphasize that if a mapping between the top 1,000 most popular applications and their corresponding identifiable ports is established, then scanning these 1,000 ports is sufficient to create highly unique fingerprints. These fingerprints would possess an entropy that exceeds 50 when combined with established fingerprinting techniques, which is theoretically enough to uniquely identify any user.

It is essential to note that these results are based on our estimates, and a large-scale real-world study would be necessary to validate these claims in a real-world setting. Nonetheless, we believe that our estimates are not implausible, thousands of ports being either open or closed is a significant characteristic that has been omitted from all existing browser fingerprinting studies, and would greatly enhance these fingerprints.


\subsubsection{Browser-based port scanning is a threat to user privacy and security}

The practice of browser-based port scanning poses a significant threat to user privacy. We argue that browser-based port scanning should be banned by web browsers, or at the very least, users should have to explicitly grant permissions for a website to access their local network. 
We contend that any potential benefits, such as the detection of remote access tools for comprised devices as a security measure, are outweighed by the risks to user privacy and security.
Additionally, we think that using browser-based port scanning as a security measure is inherently insufficient, and there are better --- less intrusive --- established alternatives such as multi-factor authentication, which can prevent hijacked devices from making unauthorized requests.

While the potential defensive mechanism of browser-based port scanning is insufficient, the threats are real.
We established the following threats to privacy in this paper:
\begin{itemize}
    \item Identification of specific open ports: Browser-based port scanning can be used to scan specific ports on a system, corresponding to an application natively opening that specific port. This knowledge can be leveraged to target users who are utilizing a particular type of application or service.
    \item Highly unique fingerprints: Browser-based port scanning, combined with existing fingerprinting techniques, can create highly unique fingerprints, effectively eliminating user anonymity on the internet.
\end{itemize}

While not a focus of this paper, the threats to security are also a significant consideration:
\begin{itemize}
    \item Drive-By Pharming~\citescientific{stamm2007}: Drive-by Pharming was an attack in which a malicious web page, when visited, attempted to change a victim's router DNS settings to gain control over their internet traffic and potentially compromise their credentials.
    \item SOHO Pharming~\citetechnical{sohopharming2013}: SOHO Pharming was an attack that compromised over 300,000 consumer-grade SOHO routers, primarily in Europe and Asia since December 2013. Attackers manipulated DNS configurations, redirected DNS requests, and performed Man-in-the-Middle attacks. The attack targeted various router models, exploited vulnerabilities such as authentication bypass and CSRF techniques, and took advantage of consumer unfamiliarity with router configurations and insecure default settings.
\end{itemize}

\subsubsection{Opt-In Solution for Enhanced Privacy and Security}

The only real benefit to accessing the local network would be a website integrating with its native desktop application. 
Granting explicit permission for this access aligns with established practices for various resources accessed by browsers, such as microphones and cameras.
The Brave browser has already implemented this functionality on both their desktop and mobile browser~\citearticle{brave_localhost}. 
We argue that more mainstream browsers such as Chrome, Firefox, Safari and Edge should adopt this change and give users more control over their privacy. 

West and Rigoudy~\citetechnical{rigoudy2023} have worked on a draft specification to prevent unauthorized private network access from the public internet. 
While this specification is still in draft and not on track to be a W3C standard, we think it should be. 
The principle of least privilege is becoming more popular in modern software architecture. 
Software developed (i.e. browsers) for end users should embrace this principle. 
By adopting the principle of least privilege, software can ensure that users are protected by default, rather than exposed to external threats by default, making it clear that the opt-in approach should be the standard for safeguarding user security and privacy.

In the current landscape, users have several options available to protect themselves against browser-based port scanning. 
One approach involves choosing browsers prioritizing user consent, like the Brave browser. 
However, transitioning to a new browser might pose usability challenges despite its emphasis on privacy.
Another strategy revolves around disabling JavaScript, a step effective in enhancing privacy, but detrimental to modern webpage functionality.
A more realistic alternative is a browser extension, such as the open source \emph{LAN Port Scan Forbidder}~\citeartifact{lanportscanforbidder}. 
This extension prevents unauthorized access to local networks. 
Nevertheless, relying on these measures highlights the conflict between privacy and usability and underscores the need for default user privacy in online environments.
Advocating for a shift where user privacy and security are default settings in modern web browsers remains crucial. 
This approach would alleviate the need for users to resort to these measures and promote a more user-centric approach to online privacy.

\clearpage


\section{Future work}

\subsubsection{Real-world validation}

As we have demonstrated that browser-based port scanning can be used to enhance existing fingerprinting methods and create unique fingerprints, a logical follow-up would be to assess its potential for user-tracking in a real world setting.
The ability to identify and track individuals across various online services and platforms has far-reaching consequences, affecting not only personal privacy but also the potential for data abuse and exploitation.

Existing fingerprinting techniques can already identify users up to a certain point. However, as demonstrated in previous chapters, browser-based port scanning can enhance the uniqueness of existing fingerprinting techniques, and can thereby increase the capabilities of user-tracking.
To evaluate the potential of browser-based port scanning in this regard, conducting a large-scale user study becomes necessary. Such a study would gauge the effectiveness of user-tracking capabilities in a real world setting, when browser-based port scanning is incorporated with existing fingerprinting techniques.

\subsubsection{Scanning private networks}

In our research, we primarily concentrated on using browser-based port scanning to scan the localhost IP address. This focus was chosen because browser-based port scanning is a time-intensive process, and the localhost IP address will generally be the most interesting IP address to scan. However, it is important to note that browser-based port scanning can be extended to scan other private IP addresses, which may reveal sensitive information.

\subsubsection{Browser-based port scanning on mobile}

Preliminary research was conducted on using browser-based port scanning to scan mobile devices. The initial findings did not yield any significant discoveries, but there exists potential for further exploration in this area, as the preliminary research was surface-level and did not dive deep into the network layer.
Mobile devices have become a primary means of accessing online services and platforms for a significant portion of the population. Understanding how browser-based port scanning behaves on mobile platforms is important, as it may have unique implications and challenges compared to desktop environments.


% \section{Discussion}
% Here, you discuss everything that might be questionable about your research. Try to be very honest, and discuss everything that makes that the results are not 100\% reliable.

% You may also make the discussion a subsection in your Conclusions.

% Check whether the best place for the discussion is before or after the Conclusions.